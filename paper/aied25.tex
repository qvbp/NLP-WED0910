% This is samplepaper.tex, a sample chapter demonstrating the
% LLNCS macro package for Springer Computer Science proceedings;
% Version 2.21 of 2022/01/12
\documentclass[runningheads]{llncs}

% 基础字体和编码
\usepackage[T1]{fontenc}

% 忽略 \vec 重定义警告
\let\vec\relax

% 数学相关包(按照依赖顺序排列)
\usepackage{amsmath}     % 最基础的数学包,应该最先加载
\usepackage{amssymb}     % 依赖于 amsmath
\usepackage{mathtools}   % 依赖于 amsmath

% 表格相关包
\usepackage{booktabs}    
\usepackage{multirow}    
\usepackage{array}       
\usepackage{float}       

% 图形和颜色
\usepackage{graphicx}
\usepackage[table]{xcolor}

% 文本修饰
\usepackage[normalem]{ulem}
\useunder{\uline}{\ul}{}

% hyperref 应该最后加载
\usepackage{hyperref}

\begin{document}
\title{HCGKT: Hierarchical Contrastive Graph Knowledge Tracing with Multi-level Feature Learning}
\titlerunning{HCGKT}

\author{Anonymous Author(s)}

\institute{Paper ID:6670}
%
\maketitle              % typeset the header of the contribution
%
\begin{abstract}
abstract
\keywords{knowledge tracing  \and hierarchical \and contrastive learning \and graph}
\end{abstract}

\section{Introduction}
\label{sec:intro}
\input{intro.tex}

\section{Related Work}
\label{sec:related}
\input{related.tex}

\section{The HCGKT Framework}
\label{sec:method}
\input{method.tex}

\section{Experiments}
\label{sec:exp}
\input{exp.tex}

\section{Conclusion}
\label{sec:conclution}
\input{conclusion.tex}

\bibliographystyle{splncs04}
\bibliography{aied}

\end{document}


